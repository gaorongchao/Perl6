% Created 2014-05-11 Sun 23:11
\documentclass{ctexart}
\usepackage[utf8]{inputenc}
\usepackage[T1]{fontenc}
\usepackage{fixltx2e}
\usepackage{graphicx}
\usepackage{longtable}
\usepackage{float}
\usepackage{wrapfig}
\usepackage{soul}
\usepackage{textcomp}
\usepackage{marvosym}
\usepackage{wasysym}
\usepackage{latexsym}
\usepackage{amssymb}
\usepackage{hyperref}
\usepackage{listings}
\usepackage{color}
\usepackage{xcolor}
\definecolor{background}{rgb}{0.9,0.9,0.9}
\tolerance=1000

\providecommand{\alert}[1]{\textbf{#1}}

\title{11: Built-in types,operators and methods}
\author{GRC(扬眉剑)}
\date{\today}
\hypersetup{
  pdfkeywords={},
  pdfsubject={},
  pdfcreator={Emacs Org-mode version 7.9.3f}}

\begin{document}

\maketitle

\setcounter{tocdepth}{3}
\tableofcontents
\vspace*{1cm}
\section{github地址}
\label{sec-1}

\href{https://github.com/gaorongchao/Perl6/tree/master/Using_perl6}{https://github.com/gaorongchao/Perl6/tree/master/Using\_perl6}

所有后续更新都在github上进行,其他地方不做同步。
发现任何错误,或者不当的地方,请先到github查看是否改正。
如果没有请留言。或者邮件交流:rongchaogao@gmail.com
\section{引入}
\label{sec-2}

很多操作符都需要一种特定类型的数据。
如果操作对象和要操作对象的类型不统一,
Perl会复制一份操作对象,然后将它转换成需要的类型。
比如:\$a+\$b 将把\$a和\$b的副本转换成数字(如果它们
已经都是数字了除外)。这种隐式的转换被称为强制转换(coercion)。

除了操作符,其他一些语法也会强制转换,if 和 while 会把数据转换成
真假(布尔值),for会把数据转换成列表,等等。
\section{数字}
\label{sec-3}

有些时候强制转换都是透明的。Perl 6 中有许多数字类型,他们都可以混合使用。
比如:用一个整数减去一个浮点数,123-12.1e1。

这里面最重要的类型是:
\begin{itemize}
\item Int
\end{itemize}
Int对象存储的是任意大小的整数。如果你如果你写的文字只包含数字,比如12,那么
他就是 Int。
\begin{itemize}
\item Num
\end{itemize}
Num是浮点型数据。它存储着符号,小数部分和指数,每一个都有固定的长度。

用科学记数法表示的数字比如 6.022e23 就是Num类型的数字。

\begin{itemize}
\item Rat
\end{itemize}
Rat是有理数的缩写,存储小数的时候不会损失精度。这是通过跟踪作为整数的
分子和分母来实现的。所以当数学运算在遇到很大的组成部分(large components)
的有理数的时候,运算速度会非常的慢。因为这个原因,当有理数有一个大分母的时候
会自动讲解为Num类型。

所以写一个以小数点为分隔符的分数值,比如3.14,就是一个Rat 类型的数。

\begin{itemize}
\item Complex 复数
\end{itemize}
复数有两个部分,实部和虚部,如果其中任何一部分为 NaN,然后整个数字也许就是 NaN。

复数是这种形式 a + bi , bi 是虚部。
\section{字符串}
\label{sec-4}
\section{布尔类型}
\label{sec-5}

\end{document}
