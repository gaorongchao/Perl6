% Created 2014-05-12 Mon 11:44
\documentclass{ctexart}
\usepackage[utf8]{inputenc}
\usepackage[T1]{fontenc}
\usepackage{fixltx2e}
\usepackage{graphicx}
\usepackage{longtable}
\usepackage{float}
\usepackage{wrapfig}
\usepackage{soul}
\usepackage{textcomp}
\usepackage{marvosym}
\usepackage{wasysym}
\usepackage{latexsym}
\usepackage{amssymb}
\usepackage{hyperref}
\usepackage{listings}
\usepackage{color}
\usepackage{xcolor}
\definecolor{background}{rgb}{0.9,0.9,0.9}
\tolerance=1000

\providecommand{\alert}[1]{\textbf{#1}}

\title{11: Built-in types,operators and methods}
\author{GRC(扬眉剑)}
\date{\today}
\hypersetup{
  pdfkeywords={},
  pdfsubject={},
  pdfcreator={Emacs Org-mode version 7.9.3f}}

\begin{document}

\maketitle

\setcounter{tocdepth}{3}
\tableofcontents
\vspace*{1cm}
\section{github地址}
\label{sec-1}

\href{https://github.com/gaorongchao/Perl6/tree/master/Using_perl6}{https://github.com/gaorongchao/Perl6/tree/master/Using\_perl6}

所有后续更新都在github上进行,其他地方不做同步。
发现任何错误,或者不当的地方,请先到github查看是否改正。
如果没有请留言。或者邮件交流:rongchaogao@gmail.com
\section{引入}
\label{sec-2}

很多操作符都需要一种特定类型的数据。
如果操作对象和要操作对象的类型不统一,
Perl会复制一份操作对象,然后将它转换成需要的类型。
比如:\$a+\$b 将把\$a和\$b的副本转换成数字(如果它们
已经都是数字了除外)。这种隐式的转换被称为强制转换(coercion)。

除了操作符,其他一些语法也会强制转换,if 和 while 会把数据转换成
真假(布尔值),for会把数据转换成列表,等等。
\section{数字}
\label{sec-3}

有些时候强制转换都是透明的。Perl 6 中有许多数字类型,他们都可以混合使用。
比如:用一个整数减去一个浮点数,123-12.1e1。

这里面最重要的类型是:
\begin{itemize}
\item Int
\end{itemize}
Int对象存储的是任意大小的整数。如果你如果你写的文字只包含数字,比如12,那么
他就是 Int。
\begin{itemize}
\item Num
\end{itemize}
Num是浮点型数据。它存储着符号,小数部分和指数,每一个都有固定的长度。

用科学记数法表示的数字比如 6.022e23 就是Num类型的数字。

\begin{itemize}
\item Rat
\end{itemize}
Rat是有理数的缩写,存储小数的时候不会损失精度。这是通过跟踪作为整数的
分子和分母来实现的。所以当数学运算在遇到很大的组成部分(large components)
的有理数的时候,运算速度会非常的慢。因为这个原因,当有理数有一个大分母的时候
会自动讲解为Num类型。

所以写一个以小数点为分隔符的分数值,比如3.14,就是一个Rat 类型的数。

\begin{itemize}
\item Complex 复数
\end{itemize}
复数有两个部分,实部和虚部,如果其中任何一部分为 NaN,然后整个数字也许就是 NaN。

复数是这种形式 a + bi , bi 是虚部。

下面所有的操作符都可以操作上面的数字类型。:

大部分数学函数都既可以写成方法调用也可以写成函数的形式,所以你既可以这样写,
(-5).abs 也可以这样写 abs(-5)。

三角函数 sin, cos, tan, asin, acos, atan, sec, cosec, cotan, asec, 
acosec, acotan, sinh, cosh, tanh, asinh, acosh, atanh, sech, cosech,
 cotanh, asech, acosech 和 acotanh 默认用弧度来计算。当然你可以自己指定一个
参数来使用 Degrees , Gradians , Circles 来进行三角函数的计算。
比如:180.sin(Degrees) 近似等于0。
\subsection{双目操作符}
\label{sec-3-1}


\begin{center}
\begin{tabular}{ll}
 **   &  乘方  \\
 $*$  &  乘法  \\
 /    &  除法  \\
 div  &  整除  \\
 \%   &  取余  \\
 +    &  加    \\
 $-$  &  减    \\
\end{tabular}
\end{center}
\subsection{单目运算符}
\label{sec-3-2}


\begin{center}
\begin{tabular}{ll}
 +    &  转换为数值  \\
 $-$  &  非 取反     \\
\end{tabular}
\end{center}
\subsection{数学函数或方法调用}
\label{sec-3-3}


\begin{center}
\begin{tabular}{ll}
 abs    &  绝对值                           \\
 sqrt   &  平方根                           \\
 log    &  自然对数                         \\
 log10  &  以10为底的自然对数               \\
 ceil   &  不小于当前数的整数               \\
 floor  &  不大于当前数的整数               \\
 round  &  四舍五入                         \\
 sign   &  正负号,-1是负数,0是0,1是正数  \\
\end{tabular}
\end{center}
\section{字符串}
\label{sec-4}

字符串存储类型为 Str ,存储的内容是一串字符,独立于字符编码之外(?)。
Buf 类型的字符串存储的是二进制数据。调用 encode 方法可以把 Str 类型转换成
 Buf 类型。 decode 把 Buf 类型转换为 Str 。

下面是对字符串的操作符:
\subsection{字符串双目操作符}
\label{sec-4-1}


\begin{center}
\begin{tabular}{ll}
 \~{}  &  连接操作符:'a'\~{}'b' 变成 `ab'  \\
 x     &  重复操作符号:'a' x 2 变成 `aa'   \\
\end{tabular}
\end{center}
\subsection{字符串单目操作符}
\label{sec-4-2}

|\~{}| 转换成字符串:\~{}1 变成 `1'   |
\subsection{字符串函数,调用方法}
\label{sec-4-3}


\begin{center}
\begin{tabular}{ll}
 .chomp                     &  去掉字符串末尾的换行符                            \\
 .substr(\$start,\$length)  &  截取一段字符,\$length 默认是剩下的字符的长度     \\
 .chars                     &  字符串中字符的数目                                \\
 .uc                        &  转换为大写                                        \\
 .lc                        &  转换为小写                                        \\
 .ucfirst                   &  把第一个字符转换为大写字符                        \\
 .lcfirst                   &  把第一个字符转换成小写                            \\
 .capitalize                &  把单词的第一个字符转换为大写,其他字符转换为小写  \\
\end{tabular}
\end{center}
\section{布尔值}
\label{sec-5}

一个布尔值要么是 True 要么是 False 。
任何值在布尔上下文中都可以转换为布尔值。
决定一个值是真还是假的规则因值的类型不同而不同:
\begin{itemize}
\item 字符串,空字符串和'0'被认为是 False。其他类型转换为 True 。
\item 数值,0为假,其他都为真
\item 列表和哈希,类型的集合比如,列表,和哈希,如果是空的,那么认为是False,反之认为是True。
\end{itemize}

if 等控制结构可以自动在布尔上下中,根据条件,计算真假。
你可以通过在表达式前添加?,来强制转换为布尔上下文。

前置的!可以用来取反,若它后面的操作数为真,就返回假。

\lstset{columns=flexible,backgroundcolor=\color{background},frame=trBL,frameround=fttt,breaklines=true,language=Perl}
\begin{lstlisting}
my $num=5;

# 隐式的布尔环境
if $num { say "True" }

# 明确的布尔环境
my $bool=?$num;
say $bool;

# 取反
my $not_num=!$num;
say $not_num;

C:\Windows\system32\cmd.exe /c perl6 a.pl
True
True
False
Hit any key to close this window...
\end{lstlisting}

\end{document}
