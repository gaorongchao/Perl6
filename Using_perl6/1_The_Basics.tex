% Created 2014-03-23 Sun 10:41
\documentclass[11pt]{article}
\usepackage[utf8]{inputenc}
\usepackage[T1]{fontenc}
\usepackage{fixltx2e}
\usepackage{graphicx}
\usepackage{longtable}
\usepackage{float}
\usepackage{wrapfig}
\usepackage{soul}
\usepackage{textcomp}
\usepackage{marvosym}
\usepackage{wasysym}
\usepackage{latexsym}
\usepackage{amssymb}
\usepackage{hyperref}
\tolerance=1000
\providecommand{\alert}[1]{\textbf{#1}}

\title{2: The Basics}
\author{GRC(扬眉剑)}
\date{\today}
\hypersetup{
  pdfkeywords={},
  pdfsubject={},
  pdfcreator={Emacs Org-mode version 7.9.3f}}

\begin{document}

\maketitle

\setcounter{tocdepth}{3}
\tableofcontents
\vspace*{1cm}
\section{about}
\label{sec-1}

github地址:\href{https://github.com/gaorongchao/Perl6/tree/master/Using_perl6}{https://github.com/gaorongchao/Perl6/tree/master/Using\_perl6}

如果发现任何错误和翻译不当的地方,请告知,非常感谢。
\section{第二章:基础}
\label{sec-2}

Perl起源于一个致力于从文本文件中收集整理信息的编程语言。
现在Perl在文本处理方面仍然强大,Perl 5 是一个一般意义上的强力编程语言。
但是Perl 6更为杰出。

假设你举办了一个乒乓球联赛。
裁判告诉你比赛结果的格式:选手1 选手2 | 3:2,这意味着,选手1赢了3局,选手2赢了2局。
你需要一个脚本来,来统计选手赢了几场比赛,赢了多少局,并以此决定最终冠军的归属。

输入文件格式如下(储存在一个名为“scores”的文件中):

\begin{verbatim}
Beth Ana Charlie Dave 
Ana Dave            |  3:0
Charlie Beth        |  3:1
Ana Beth            |  2:3
Dave Charlie        |  3:0
Ana Charlie         |  3:1
Beth Dave           |  0:3
\end{verbatim}

第一行是所有选手的名字,后面跟着的是每场比赛的结果。
这里有一个解决此问题的Perl 6 脚本:perl-1-1:

\begin{verbatim}
 1:  use v6;
 2:  
 3:   my $file = open 'scores';
 4:   my @names = $file.get.words;
 5:  
 6:   my %matches;
 7:   my %sets;
 8:  
 9:   for $file.lines -> $line {
10:   my ($pairing, $result) = $line.split(' | ');
11:   my ($p1, $p2) = $pairing.words;
12:   my ($r1, $r2) = $result.split(':');
13:  
14:   %sets{$p1} += $r1;
15:   %sets{$p2} += $r2;
16:  
17:   if $r1 > $r2 {
18:   %matches{$p1}++;
19:   } else {
20:   %matches{$p2}++;
21:   }
22:   }
23:  
24:   my @sorted = @names.sort({ %sets{$_} }).sort({ %matches{$_} }).reverse;
25:  
26:   for @sorted -> $n {
27:   say "$n has won %matches{$n} matches and %sets{$n} sets";
28:   }
\end{verbatim}
输出文件:

\begin{verbatim}
Ana has won 2 matches and 8 sets
Dave has won 2 matches and 6 sets
Charlie has won 1 matches and 4 sets
Beth has won 1 matches and 4 sets
\end{verbatim}
每一个Perl6程序都需要以 “use v6”; 开始。这一行告诉编译器Perl代码的版本。
如果,你不小心用Perl5来运行这个程序,你将得到错误信息。

一个Perl6程序包含0行或者更多的语句。每一个语句以分号结尾,或者以大括号结尾:

\begin{verbatim}
1:  my $file = open 'scores';
\end{verbatim}
``my''声明一个词法变量。词法变量只在当前代码块可用,从声明开始到代码块的结束。
如果一个代码块始终没有闭合,那么该变量可以在剩下的所有地方使用。代码块的定义是:
一段在花括号\{\}之间的代码。

一个变量名称以“魔符”开始,魔符是指那些非字母,非数字的符号,比如:\$,@,\% 或者\&-
或者有时出现的双冒号“::”也是。
魔符指明了变量的结构类型,比如:它应该被当作一个值,还是多个值,还是一个子程序,等等。
跟在“魔符”后面的是“标识符”,标识符可以包含字母,数字和下划线。在字母与字母之间,
你可以用短横线“-”或者撇号“'”。所以“isn't”和“double-click”都是合法的变量标识符。

魔符“\$”表明变量是一个数值变量,指明这个变量只存储了一个值。

内建函数“open”打开了一个名为“scores”的文件,然后返回一个文件句柄,
文件句柄是一个代表该文件的对象。
赋值符号“=”,把文件句柄赋值给左边的变量,这也就是意味着变量\$file现在存储着文件句柄。

“scores”是字符串文本(string literal)。字符串(string)是一段纯文本。
字符串文本(string literal)是直接出现在程序中的字符串。
在这一行中,它是提供给open函数的参数。


\begin{verbatim}
1:  my @names = $file.get.words;
\end{verbatim}
上面语句的右侧调用了一种“方法”(一个命名的一系列行为的集合体)获取存储在\$file变量中的句柄。
被命名为get的这一方法从文件中读取并返回一行,去掉末尾的换行符。word同样是一个方法,调用从
get中返回的一行文本。word这一方法,分解它的invocant(它要操作的字符串)成一系列的单词。
这里单词的意思是不包含空格的字符串。它把字符串“Beth Ana Charlie Dave”分解成多个字符串
“Beth”,“Ana”,“Charlie”,“Dave”。

最后,这一字符串列表存储到数组@names中。魔符@表明所声明的变量是一个数组。数组存储的是有序列表。


\begin{verbatim}
1:  my %matches;
2:  my %sets;
\end{verbatim}
这两行代码声明了两个哈希。魔符“\%”把变量标记为哈希。哈希又称为散列:是无序的键-值对的集合。
在其他语言中又称为“哈希表”,“字典”或者“(map)图”
你可以通过“键 \$key”利用“\%hash\{\$key\}”来查询在哈希表中的值。

在上述计分程序中,\%matches记录了每一位运动员赢的次数。\%sets记录了每一位运动员赢的局数。

魔符指出了默认访问变量的方法。数组变量@是通过位置来访问,
哈希变量\%是通过“键”来访问。
数值变量\$表示一个大箩筐,这里可以盛任何东西,也可以用任何方式来访问。
一个数值变量可以包含一个复杂对象(compound object),比如:数组或者哈希;
魔符\$表明它应该被当作一个单独的值,尽管有可能它包含众多的值(像一个数组或者哈希)。

\begin{verbatim}
1:  for $file.lines ->$line{
2:  ...
3:  }
\end{verbatim}
“for”产生了一个由花括号界定运行范围的循环,循环会遍历列表中的每一个值。
\$file.lines 从scores文件中读取了很多行,除掉前面\$file.get读取的那一行以外,到最后一行。
循环依次将每一行的值赋予\$line变量。

第一次迭代\$line将包含字符串 “Ana Dave | 3:0”;第二次迭代\$line将包含字符串“Charlie Beth | 3:1”,等等。


\begin{verbatim}
1:  my ($pairing,$result) = $line.split('|');
\end{verbatim}
“my”可以同时声明多个变量。在赋值符号(=)的右侧调用了一个名为“split”的方法,它把“|”当作参数。

split把他作用的内容以|为分割点,分割成一系列的字符串。所以如果你用“|”作为粘合符号,把这些字符串粘合。
那么你将得到原始字符串。

\$pairing 得到返回列表的第一个元素,\$result 得到第二个。

当处理完第一行以后,\$pairing 包含“Ana Dave”这个字符串,\$result包含“3:0”。

下面两行是同样的模式:

\begin{verbatim}
1:  my ($p1,$p2) = $pairing.words;
2:  my ($r1,$r2) = $result.split(':');
\end{verbatim}
第一行代码提取并保存两位运动员的姓名到\$p1和\$p2这两个变量中。

第二行提取每一位运动员的比赛结果,并且分别保存到\$r1和\$r2中。

上面程序运行以后,下面列出每一个变量包含的值:


Table2.1:Contents of Variables

\begin{verbatim}
1:  Variable  Contents               
2:  $line     'Ana Dave    | 3:0' 
3:  $pairing  'Ana Dave'              
4:  $result   '3:0'                   
5:  $p1       'Ana'                   
6:  $p2       'Dave'                  
7:  $r1       '3'                     
8:  $r2       '0'
\end{verbatim}

程序接下来统计了每一位运动员赢的局数:

\begin{verbatim}
1:  %sets{$p1} += $r1;
2:  %sets{$p2} += $r2;
\end{verbatim}
这是下面的简写:

\begin{verbatim}
1:  %sets{$p1} = %sets{$p1}+$r1;
2:  %sets{$p2} = %sets{$p2}+$r2;
\end{verbatim}
+= \$r1 的含义是:把左侧的变量加上\$r1.
在第一次迭代的时候\%sets\{\$p1\}还没有设置类型,
所以默认为一个称为“Any”的特殊值。
加法和递增操作符作为一个数字操作符,把Any当作一个值为0的数字。
所以这个字符串自动转换为数字。

在这两行代码执行以前,\%sets 是空的。
在hash中添加一个完全不存在的元素,将会使这个元素立马在哈希中存在。
并且赋予初始值为0.(这被称为autovivification)。
当这两行第一次运行以后,\%sets包含“ `Ana'=> 3,'Dave'=>0”。
(胖箭头=>把键值对分开)


\begin{verbatim}
1:  if $r1>$r2{
2:      %matches{$p1}++;
3:  } else {
4:      %matches{$p2}++;
5:  }
\end{verbatim}
如果\$r1 的值大于\$r2的值,那么\%matches\{\$p1\}增加1。
如果\$r1 的值小于\$r2的值,那么\%matches\{\$p2\}增加1。
和前面的+=的例子一样,如果哈希中没有存在这个值,
那么这个哈希值将会因为自增符号而自动生成。

\$thing++ 是\$thing +=1的缩写。
需要注意的一点是,\$thing++这个表达式返回的\$thing的值是没有增加前的值。
而不是增加以后的。在其他很多的编程语言中你可以把++前置。这样得到的增加
以后的返回值。my \$x=1; say ++\$x 输出2。


\begin{verbatim}
1:  my @sorted = @names.sort({ %sets{$_} }).sort({ %matches{$_} }).reverse;
\end{verbatim}

这一行代码包含三个独立的过程。首先调用数组的sort方法。
但是默认的排序方法是按照内容排序的。为了按照赢得多少的顺序输出运动员的姓名,
我们的程序必须按照运动员的成绩来排序,而不是用他们的名字。
sort方法的参数一个代码块,用来把数组元素(运动员的名字)转化为用来排序的数据。
数组元素是通过“\$\_”变量来传递的。

你在前面已经看到过代码块:不论是“for loop-> \{...\}”还是在代码块中的if语句。
代码块是:一个独立的带有 signature(the ->\$line 的部分)Perl6代码。
更多内容请参照sec::signatures。

按照运动员成绩排序的最简单的方法是@names.sort(\{ \%matches\{\$\_\} \}),
这是按照运动员赢的次数决定的。但是问题是,Ana 和Dave都赢了2次。
最简单的排序方式并没有考虑到每个人赢的局数,而这个正是决定谁赢的循环赛的第二个评判标准。

如果两个数组元素拥有相同的值, sort按照发现他们的先后顺序放置他们。
计算机科学家们称之为:稳定排序。
这个程序充分利用了Perl6排序的相关特性来利用两次排序来实现目标:
第一次按照赢取局数的多少排序(也就是次要的冠军评判标准),然后按照赢取的次数来排序。

在第一次排序以后,运动员的名字顺序是这样的:Beth Charlie Dave Ana。
在第二次排序以后,顺序依然一样。因为没有人赢了更少的次数,但是赢了更多的局数。
这种情况是很正常的,特别是在大型的循环赛中。

sort是按照升序来排序,也就是从小到大排序。
但这与我们的期望是相反的。所以,我们的程序在第二次排序以后,又调用了.reverse方法。
最后把结果存入到@sorted数组中。


\begin{verbatim}
1:  for @sorted -> $n {
2:      say "$n has won %matches{$n} matches and %sets{$n} sets";
3:  }
\end{verbatim}

为了输出运动员的名字和他们的成绩,我们对@sorted数组进行循环。
依次将运动员名字赋值给\$n。我们可以这样来读代码“对数组sorted中的每一个元素,
赋值给\$n,然后执行下面的代码块”。
然后用say 进行标准输出(也就是输出到屏幕)。
然后后面自动加上换行符。
如果你不想在最后加上换行符,那就用print函数。

当你运行本程序的时候,你会发现,say并不会逐字的输出所有内容。
在\$n的位置上,它将会打印变量\$n的内容:也就是存储在\$n中的运动员的名字。
这种自动替换被称为“interpolation(变量内插)”
这种变量内插只会发生在双引号内,而不会发生在单引号内。

\begin{verbatim}
1:  my $names = 'things';
2:  say 'Do not call me $name'; # Do not call me $names
3:  say "Do not call me $name"; # Do not call me things
\end{verbatim}
在Perl6中,双引号不仅能够内插魔符\$的变量,同时也可以内插在花括号内的代码块。
因为任意的Perl代码都可以通过花括号出现,所以可以通过把数组和哈希放在花括号内
实现内插的效果。

数组放在花括号内插以后,两个元素之间会插入一个空格。
哈希放在花括号内插以后,以多行的形式显示,每一行包括一个键,然后跟着一个制表符。
然后紧跟这键对应的值,最后加上换行符。

\begin{verbatim}
1:  say "Math: {1+2}";                  # Match: 3
2:  my @people = <Luke Mattew Mark>;
3:  say "The synoptics are: {@people}"; # The synoptics are :Luck Matthew Mark
4:  say "{%sets}";                      # 接前面网球循环赛
5:  # Charlie 4
6:  # Dave    6
7:  # Ana     8
8:  # Beth    4
\end{verbatim}
当数组和哈希变量直接出现在双引号之间的时候(并没有在花括号之间),那么只有在它们的名字
后面紧跟方括号的时候才会变量内插。当然你也可以在变量名和postcircumfix之间调用一种
方法来实现变量内插。

\begin{verbatim}
 1:  my @flavours = <vanilla peach>;
 2:  say "we have @flavours";          # we have @flavours
 3:  say "we have @flavours[0]"        # we have vanilla
 4:  # so-called "Zen slice"
 5:  say "we have @flavours[]";        # we have Vanilla peach 
 6:  
 7:  # method calls ending in postcircumfix
 8:  say "we have @flavours.sort()";   # we have peach  vanilla
 9:  
10:  # chained method calls:
11:  say "we have @flavours.sort.join(',')"; # we have peach ,vanilla
\end{verbatim}

\end{document}
