% Created 2014-05-11 Sun 09:15
\documentclass{ctexart}
\usepackage[utf8]{inputenc}
\usepackage[T1]{fontenc}
\usepackage{fixltx2e}
\usepackage{graphicx}
\usepackage{longtable}
\usepackage{float}
\usepackage{wrapfig}
\usepackage{soul}
\usepackage{textcomp}
\usepackage{marvosym}
\usepackage{wasysym}
\usepackage{latexsym}
\usepackage{amssymb}
\usepackage{hyperref}
\usepackage{listings}
\usepackage{color}
\usepackage{xcolor}
\definecolor{background}{rgb}{0.9,0.9,0.9}
\tolerance=1000

\providecommand{\alert}[1]{\textbf{#1}}

\title{1: Preface}
\author{GRC(扬眉剑)}
\date{\today}
\hypersetup{
  pdfkeywords={},
  pdfsubject={},
  pdfcreator={Emacs Org-mode version 7.9.3f}}

\begin{document}

\maketitle

\setcounter{tocdepth}{3}
\tableofcontents
\vspace*{1cm}
\section{github地址}
\label{sec-1}

\href{https://github.com/gaorongchao/Perl6/tree/master/Using_perl6}{https://github.com/gaorongchao/Perl6/tree/master/Using\_perl6}

所有后续更新都在github上进行,其他地方不做同步。
发现任何错误,或者不当的地方,请先到github查看是否改正。
如果没有请留言。或者邮件交流:rongchaogao@gmail.com
\section{第一章:前言}
\label{sec-2}

Perl6是这样一门编程语言,在整个程序的完整周期中,在各个不同的
阶段有着不同的编译器和解释器。
这些东西的实现又反过来影响了这门语言的设计,
帮助发现那些性能不足、矛盾或者难以实现且得不偿失的特性。
(by highlighting misfeatures,contradictions,or features
of difficult implementation and little benefit)
这门语言通过反复的改进,成为一个衔接顺畅且统一的规范语言。
(翻译待改善)。

Perl6是一门通用,直观灵活的语言。它融合了如程序化(procedural),
面向对象,和函数式编程等众多范式;同时也为文本解析提供了强大的工具。

这本书正在书写中。已经发布的版本中依然包许多TODO的注释,这些是我们在
正式刊印本书前要做的。我们保留TODO是因为他们可以为读者和我们提供一个有用
的提示,让我们知道那些地方需要改善和完成。
尽管如此,我们仍然要祈求读者的宽容和理解。
\subsection{Audience 读者人群}
\label{sec-2-1}

这本书主要是为那些想学习Perl6的人准备的。
它只是一系列的学习指南,而不是综合的参考手册。
尽快不要求对Perl有一定的认识,但是我们依然希望读者能够有一些使用其他编程语言的
经验。
\subsection{本书的结构}
\label{sec-2-2}

每一章以一个合理而完整的例子来阐述本章的主题。
我们希望通过解释每一章中真实的程序,能够通过这些例子来展现如何使用这些特色,技巧和惯用语。
我们的目标是把Perl6的特点在程序中表现出来,这样读者就可以自己写出地道的Perl6程序,
而不是写出像其它语言的程序。
\subsection{Perl6和Perl5的关系}
\label{sec-2-3}

Perl6是Perl大家族中新的一员。从5到6的进化中,Perl6在语法和语义的兼容性
上有了很大的代沟。但是这并不意味着Perl5会消逝。实际上,恰恰相反,Perl5和
Perl6都有活跃的开发团队,正是他们铸造了这门语言。
Perl5的开发人员,都在保持兼容老版本Perl的基础上,用不同的方式来进行扩展。
而Perl6的开发人员,通过添加新的语法和语义上功能来让Perl这门语言更加的强大和
有更强的表现能力,而不用严格的考虑兼容Perl5以及以前的版本。

也许有些人会问,“既然是一门不同的语言为何还称之为Perl?”Perl不仅仅是一种语法。
而是一种理念(比如:做事情的方法不只有一种;让容易的事情更容易,不容易的事情可以实现)
;还是一种风俗(comprehensive testing,idioms);还是一座大厦(CPAN);还是一个社群
(perl5-porters,perl6-language)。不论Perl5还是Perl6都在不同程度上烙有这些基因。
同样,Perl也是包容的。如同Perl从其它语言借鉴好的想法一样,Perl5和Perl6也共享着一些特性。
\subsection{Perl 6 的实现}
\label{sec-2-4}

Perl 6是一种规范。任何一种能完整通过官方测试的实现方式都可以把自己称之为Perl6。
现在已经存在几种不同的实现形式,他们的完备性各不相同。本书的所有例子都是在
Rakudo Perl 6 编译器上运行,但是这并不是说他们只能在Rakudo上才能运行,任何一个
成熟的Perl 6的实现都可以运行这些例子。最后祝您好运,如同Perl 6 团队经常说的-玩的开心。
\subsection{安装Rakudo}
\label{sec-2-5}

对于Rakudo的下载和安装的详细说明,请看:\href{http://www.rakudo.org/how-to-get-rakudo}{http://www.rakudo.org/how-to-get-rakudo}.
源代码的版本:\href{http://github.com/rakudo/rakudo/downloads}{http://github.com/rakudo/rakudo/downloads} 。
一个windows可用的二进制版本可以从这里下载:\href{http://sourceforge.net/projects/parrotwin32/files/}{http://sourceforge.net/projects/parrotwin32/files/} 。

Rakudo star 从这里下载:\href{http://rakudo.org/downloads/star/}{http://rakudo.org/downloads/star/}
\subsection{执行程序}
\label{sec-2-6}

要用Rakudo来运行Perl 6程序,你需要在你的系统路径中添加安装Rakudo的路径。
并在命令行中执行这样的命令:

\lstset{columns=flexible,backgroundcolor=\color{background},frame=trBL,frameround=fttt,breaklines=true,language=Perl}
\begin{lstlisting}
$ perl6 hello.pl
\end{lstlisting}
如果你调用Perl6编译器的时候没有指定明确的脚本,那么它将进入一个交互式模式,
允许你在命令行中执行Perl6的语句。
\subsection{参与进来}
\label{sec-2-7}

如果你受到本书内容的启发,并且萌发了对Perl6社区做一些贡献的欲望,那么这里
有一些有用的资源提供给你:
\begin{itemize}
\item World Wide Web 万维网
\end{itemize}
Perl 6 的主页:\href{http://perl6.org/}{http://perl6.org/} 这里链接了很多有用的资源。
\begin{itemize}
\item IRC
\end{itemize}
irc.freenode.net 中的\#perl6频道讨论Perl6的各个方面。
\begin{itemize}
\item 邮件列表
\end{itemize}
如果你需要在Perl6编程方面得到帮助,请发送邮件到 perl6-users@perl.org 。

对Perl6语言规范的问题,请发送邮件到 perl6-language@perl.org 。

对Perl6编译器的问题,请发送邮件到perl6-compiler@perl.org 。

\end{document}
