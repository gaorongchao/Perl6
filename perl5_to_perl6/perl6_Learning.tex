% Created 2014-05-10 Sat 11:30
\documentclass{ctexart}
\usepackage[utf8]{inputenc}
\usepackage[T1]{fontenc}
\usepackage{fixltx2e}
\usepackage{graphicx}
\usepackage{longtable}
\usepackage{float}
\usepackage{wrapfig}
\usepackage{soul}
\usepackage{textcomp}
\usepackage{marvosym}
\usepackage{wasysym}
\usepackage{latexsym}
\usepackage{amssymb}
\usepackage{hyperref}
\usepackage{listings}
\usepackage{color}
\usepackage{xcolor}
\definecolor{background}{rgb}{0.9,0.9,0.9}
\tolerance=1000

\providecommand{\alert}[1]{\textbf{#1}}

\title{perl6学习}
\author{扶 凯}
\date{\today}
\hypersetup{
  pdfkeywords={},
  pdfsubject={},
  pdfcreator={Emacs Org-mode version 7.9.3f}}

\begin{document}

\maketitle

\setcounter{tocdepth}{3}
\tableofcontents
\vspace*{1cm}
\section{Github地址}
\label{sec-1}

\href{https://github.com/gaorongchao/Perl6/tree/master/perl5_to_perl6}{https://github.com/gaorongchao/Perl6/tree/master/perl5\_to\_perl6}
所有后续更新都在github上进行,其他地方不做同步。
发现任何错误,或者不当的地方,请先到github查看是否改正。
如果没有请留言。或者邮件交流:rongchaogao@gmail.com
\section{写在perl6之前}
\label{sec-2}

从95年到现在,很多年过去了,从perl6计划开始到现在,也有10年的时间了。
所有的事物都在不断的发展,或许perl6还要再开发10年,或者20年,但是未来
总是属于perl6的。与其等到40岁的时候再去学习一项新的东西,不如从现在开
始就积极拥抱perl6。
Perl6现在的版本Rakudo每个月都在更新新的版本。分为两种,一个是Rakudo,
一个是Rakudo star。大家可以选择安装。

我以前只翻译到第三部分,后面就没有翻译下去。扶凯兄基本上都翻译完成。
本翻译是在扶凯兄翻译的基础上,加工整理而成。
\section{perl6的安装}
\label{sec-3}

perl6的官方网站:perl6.org。上面有perl6相关的东西。其中Compilers
是Rakudo,也就是perl6的编译器。在rakudo.org里面我们可以下载。或者我
们可以用git来进行安装。在下载下来的rakudo中有一个install的文件,这里
包含了安装rakudo所需要的东西,和安装的信息,里面用Ubuntu作为例子,你
可以根据这个来进行安装。

同时现在提供Rakudo star 的安装包。为windows提供了MSI安装包。
\href{http://rakudo.org/downloads/star/}{http://rakudo.org/downloads/star/}

安装完成以后,修改一下环境变量,vim .bashrc,然后添加perl6所在的目录
就可以了。
\section{学习perl6}
\label{sec-4}

我看的第一份资料是是从perl6.org/documentation/ 里下载的Perl 5 to 
perl 6.还有doc.perl6.org里面的的东西。
首先perl6中不用use strict;和use warnings;了,用了反而会报错。
\subsection{字符串,数组和哈希}
\label{sec-4-1}

用法

\lstset{columns=flexible,backgroundcolor=\color{background},frame=trBL,frameround=fttt,breaklines=true,language=Perl}
\begin{lstlisting}
 1:  # 首先不需要use strict;use warnings;
 2:  
 3:  my $five =5 ;
 4:  print " $five\n";
 5:  say "five is $five";
 6:  # 上面是一样的效果,say 自带换行符
 7:  
 8:  my @array =1,2,3,'foo';
 9:  my $sum=@array[0]+@array[1];
10:  say "sum is $sum";
11:  # 数组的元素可以这样写了。
12:  if $sum>@array[2]
13:  {
14:      say "not executed";
15:  }
16:  # if 语句这样写了,不用括号了
17:  my $number_of_elems=@array.elems;        
18:  # 下面这样也可以
19:  my $number_of_elems=+@array;
20:  say "number of elems $number_of_elems";
21:  # 数组元素的个数,不在支持$# 了。
22:  
23:  my $last_item=@array[*-1];
24:  say "last item is $last_item";
25:  # 数组的最后一个元素
26:  
27:  my %hash= foo=>1,bar=>2,baz=>3;
28:  say "hash bar is  %hash{'bar'}";
29:  say "hash bar is  %hash<bar>";# 和上面一样的效果
30:  # %hash{bar} 这样是错误的。
31:  # 这是要去找一个叫bar()的子程序,很显然我们这里没有定义。
\end{lstlisting}
perl6和perl5类似,只不过更好而已。一句话是由分号终止,
在一个语句块,或者大括号之间代码的最后一行,分号是可加,可不加的。
变量仍然由\$,@,\%等打头。许多perl5的功能依然健在。
\subsubsection{字符串strings}
\label{sec-4-1-1}

字符串被双引号包括(变量内插)或者用单引号,反斜杠和perl5一样。
变量内插的规则有了一些变化,下面的情况将被变量内插

\lstset{columns=flexible,backgroundcolor=\color{background},frame=trBL,frameround=fttt,breaklines=true,language=Perl}
\begin{lstlisting}
 1:  my $scalar = 6;
 2:  my @array =1,2,3;
 3:  say "Perl scalar is $scalar";
 4:  say "array数组的所有元素 @array[]";       
 5:  #空索引,结果是整个数组,1,2,3
 6:  say "array 数组的第二个元素 @array[1]";
 7:  say "没有索引,不会显示数组的名称: @array"; 
 8:  # 打印@array这几个字符
 9:  say "双引号内执行代码:Code:{$scalar*2}";  
10:  #大括号中的部分会被看成代码,结果是内插的,结果是12
\end{lstlisting}
数组和hash只有在后面跟着索引的时候,或者跟着以()结尾的方法的时候才会变量内插。
如”some \$obj.method()''),空的索引内插整个hash或者数组。
\subsubsection{数组}
\label{sec-4-1-2}

数组变量仍然以@开头,但是不同的是即使取其中的单个元素也是以@开头的。

\lstset{columns=flexible,backgroundcolor=\color{background},frame=trBL,frameround=fttt,breaklines=true,language=Perl}
\begin{lstlisting}
1:  my @a =5,1,2;                         
2:  #列表是用逗号来分割构建的,不用括号了。
3:  say "数组的所有元素 @a[]";
4:  say "数组的第一个元素 @a[0]";            # 不再是$a[0]
5:  say "数组的第一个和第三个元素 @a[0,2]";   # 数组切片仍然可以用
\end{lstlisting}
列表是用逗号分割构建的。1,是一个列表,(1)不是。

由于现在一切都可以看成对象,你可以用数组的方法来对数组进行操作。

\lstset{columns=flexible,backgroundcolor=\color{background},frame=trBL,frameround=fttt,breaklines=true,language=Perl}
\begin{lstlisting}
 1:  my @a =5,1,2;             #列表是用逗号来分割构建的。
 2:  say "数组a @a[]";
 3:  
 4:  my @b=@a.sort;           #排序后的数组a
 5:  say "排序后的数组 @b[]";
 6:  my $num_of_array_b=@b.elems;
 7:  say "b的元素个数 $num_of_array_b";
 8:  if @b>2 {say "如果元素个数大于2,yes"}
 9:  
10:  my $end_of_array_b=@b.end;
11:  say "b的最后一个元素的索引, $end_of_array_b";#替代了$#
12:  my @c=@b.map({$_*2});    # map 同样还是一种方法。
13:  say "数组c @c[]";
\end{lstlisting}
这里有一种qw/../的简写形式:

\lstset{columns=flexible,backgroundcolor=\color{background},frame=trBL,frameround=fttt,breaklines=true,language=Perl}
\begin{lstlisting}
1:  my @methods = <shift unshift push pop end delete sort map>;
\end{lstlisting}
数组相关新方法

\lstset{columns=flexible,backgroundcolor=\color{background},frame=trBL,frameround=fttt,breaklines=true,language=Perl}
\begin{lstlisting}
1:  @array.keys();       #取得数组的下标
2:  @array.values();     #取得数组的值
3:  @array.kv();         #下标和值一起取得
4:  @array.elems();      #元素的个数
5:  @array.exists(num); #判断某个下标值是否存在,
6:  @array.max();       #最大值
7:  @array.min();       #最小值
\end{lstlisting}


\lstset{columns=flexible,backgroundcolor=\color{background},frame=trBL,frameround=fttt,breaklines=true,language=Perl}
\begin{lstlisting}
1:  @array.pick(num);   #随机取出数组中的不重复元素
2:  #例子
3:  my @array = <a b c d e>;
4:  say @array.pick(4);
5:  say @array.pick(*);#*代表所有元素
\end{lstlisting}
\subsection{hash哈希}
\label{sec-4-2}

Perl 5 中的哈希在列表环境中依然是一个列别。
但是Perl 6 在列表环境中是列表对。列表对在其他方面也有广泛应用,
比如,子程序中的具名参数。后面会有更多的应用。
如同数组一样,哈希也有不同的调用方法。

\lstset{columns=flexible,backgroundcolor=\color{background},frame=trBL,frameround=fttt,breaklines=true,language=Perl}
\begin{lstlisting}
1:  my %drinks =
2:      France  => 'Wine',
3:      Bavaria => 'Beer',
4:      USA     => 'Coke';
5:  say "The people in France love ",%drinks{'France'};
6:  my @countries = %drinks.keys.sort;
7:  #%drinks{'France'};现在的用法
8:  #$drinks{France};  以前的用法
9:  #%drinks<France>;  现在也可以这样用
\end{lstlisting}
注意:当你访问hash的元素的时候\%hash\{...\},键并不会自动添加引号,
\%hash\{foo\}不是去访问foo的值,而是调用名称为foo()的子程序。自动quoting
并没有消失,只不过换了一种方式:

\lstset{columns=flexible,backgroundcolor=\color{background},frame=trBL,frameround=fttt,breaklines=true,language=Perl}
\begin{lstlisting}
1:  say %drinks<Bavaria>;
\end{lstlisting}
所有的内建方法,可以是method也可以是一个子程序,所以这两种方法
都对,sort @array 或者 @array.sort.\\
最后你要知道,所有的[..] \{..\}(occurring direct after a term)
仅仅是在使用一个特定的方法,而不是和数组和哈希绑定的。这意味着,他们并不在依赖于特殊的魔符。


\lstset{columns=flexible,backgroundcolor=\color{background},frame=trBL,frameround=fttt,breaklines=true,language=Perl}
\begin{lstlisting}
1:  my $a = [1,2,3];
2:  say $a[2];      #3
3:  #this implies that you don't need special dereferencing syntax,and that you can
4:  #act as arrays, hashes and subs at the same time.没整明白???
\end{lstlisting}
从这里,我们可以看到,以后不用特殊的解引用的方法,并且您可以创建能同时充当数组,哈希和子程序的
对象。
\subsection{Types 类型}
\label{sec-4-3}

概要:

\lstset{columns=flexible,backgroundcolor=\color{background},frame=trBL,frameround=fttt,breaklines=true,language=Perl}
\begin{lstlisting}
my Int $x=3;
$x="foo";   # 这里会报错,原因是,我们上面一句把$x 
# 定义为Int也就是整数,但是这里我们却给他复制一个字符串
say $x.WHAT; # 'Int()'

# 检查一个变量的类型
if $x~~Int
{
    say '$x contains an Int';
}
\end{lstlisting}
在perl6中有了类型,所有的东西都可以看成一个对象,都有一个类型。
变量也可以有一个类型的约束,但是,不是必须需要一个类型。

\lstset{columns=flexible,backgroundcolor=\color{background},frame=trBL,frameround=fttt,breaklines=true,language=Perl}
\begin{lstlisting}
1:  'a string'  #str字符串
2:  2           #int整数型
3:  3.14        #Rat (rational number)有理数
4:  (1,2,3)   #Seq列表
\end{lstlisting}
所有的内建类型都是大写字母开头,所有的标准类型都是继承了Any,并且所有的都继承了Mu


\lstset{columns=flexible,backgroundcolor=\color{background},frame=trBL,frameround=fttt,breaklines=true,language=Perl}
\begin{lstlisting}
#你可以在声明的时候,加上类型
my Numeric $x = 3.4;
my $Int @a = 1,2,3;
# 试图把一个值赋予一个错误的类型会提示出错

#对一个数组类型的类型定义,作用在数组的元素上,
#Str @s  @s这个数组只能够包含字符串元素的数组
\end{lstlisting}
一些类型是隶属与一个大的分类,比如:整数型(Int),有理数(Rat),浮点型(Num)都是属于
Numeric这个大的类型


\lstset{columns=flexible,backgroundcolor=\color{background},frame=trBL,frameround=fttt,breaklines=true,language=Perl}
\begin{lstlisting}
#要知道一个对象的类型,可以使用.WHAT的方法
say "foo".WHAT;
#如果你要确定一个某变量是不是某一特殊类型,
#这里有一个一个不同的方法,这种方法把继承考虑在内
#在这里我们推荐使用这种方法
if $x ~~ Int
{
    say 'Variable $x contains an integer';
}
\end{lstlisting}
虽然这种类型系统让我们很难彻底领悟它的所有细节。但是我们依然有很多理由使用它。

我们需要类型的原因:
\begin{itemize}
\item 1.编程更加安全
\end{itemize}
如果你声明了一个特殊的类型,那么你可以执行特定的某种操作,而不用检查
\begin{itemize}
\item 2.可优化
\end{itemize}
如果在编译的时候提供了类型,那么运行的程序会有明显的优化。在原则上perl6
不会比C慢
\begin{itemize}
\item 3.可扩展型
\end{itemize}
有了类型信息和多重的操作路径,你可以很容易对特定的类型改善操作
\subsection{基本的控制结构}
\label{sec-4-4}
\subsubsection{概要}
\label{sec-4-4-1}


\lstset{columns=flexible,backgroundcolor=\color{background},frame=trBL,frameround=fttt,breaklines=true,language=Perl}
\begin{lstlisting}
 1:  my $percent=120;
 2:  if $percent >100
 3:  {
 4:      say "weird mathematics";
 5:  }
 6:  
 7:  for 1..3
 8:  {
 9:      #用$_作为默认循环的变量
10:      say 2*$_;
11:  }
12:  
13:  for 1..3 -> $x
14:  {
15:      # 用一个明确的循环变量
16:      say 2*$x;
17:  }
18:  
19:  while $stuff.is_wrong
20:  {
21:      $stuff.try_to_make_right;
22:  }
23:  die "Access denied" unless $password eq "Secret";
\end{lstlisting}
perl6 和perl5 的控制结构基本相同,最大的不同是你不必在if,while,for等,后面添加
小括号了。事实上,所有的标示符后面紧跟着小括号,都会被看成在调用一个名称为if的子程序,
for后面加一个空格,可以改善这种情况,但是直接省略括号更加安全。
\subsubsection{分支}
\label{sec-4-4-2}

\begin{itemize}
\item if控制结构:if结构是变化最小的,你依然可以用elsif和else,
\end{itemize}
unless也还在,但是在unless后面不允许else分支结构

\lstset{columns=flexible,backgroundcolor=\color{background},frame=trBL,frameround=fttt,breaklines=true,language=Perl}
\begin{lstlisting}
1:  if $sheep == 0 {say "how boring";}
2:  elsif $sheep ==1 {say "one lonely sheep";}
3:  else {say "a herd,How lovely!";}
\end{lstlisting}
你现在依然可以使用if和unless作为语句的修饰,也就是后声明的方式:

\lstset{columns=flexible,backgroundcolor=\color{background},frame=trBL,frameround=fttt,breaklines=true,language=Perl}
\begin{lstlisting}
say "you won" if $answer == 42;
\end{lstlisting}
\begin{itemize}
\item Loops:和perl5一样,你依然可以通过next和last来控制循环。
\end{itemize}

在这里for循环只用于遍历列表,默认的变量依然是\$\_,同时你也可以显式的
声明一个循环变量。

\lstset{columns=flexible,backgroundcolor=\color{background},frame=trBL,frameround=fttt,breaklines=true,language=Perl}
\begin{lstlisting}
1:  for 1..100 -> $x
2:  {
3:      say $x;#会输出1 2 3 。。
4:  }
\end{lstlisting}
->\$x\{..\}被称为‘pointy block'如同匿名子程序或者lisp中的lambda。

当然,也可以有不止一个的循环变量。

\lstset{columns=flexible,backgroundcolor=\color{background},frame=trBL,frameround=fttt,breaklines=true,language=Perl}
\begin{lstlisting}
for 0..5 ->$even,$odd
{
    say "Even: $even \t Odd: $odd";
}
#结果如下:
#Event:1     Odd:2
#Event:3     Odd:4
#也就是交替出现

#这个也是遍历哈希的方法
my %hash=
    a   => 1,
    b   => 2,
    c   => 3;
for %hash.kv -> $key,$value
{
    say "$key: $value";
}

#C-风格的for循环,唯一需要括号的循环结构
loop (my $x=1;$x<100;$x**2)
{
    say $x;
}
\end{lstlisting}
\subsection{Subroutines and Signatures 子程序和参数}
\label{sec-4-5}
\subsubsection{语法}
\label{sec-4-5-1}

\begin{itemize}
\item Perl5 样式的子程序
\end{itemize}

\lstset{columns=flexible,backgroundcolor=\color{background},frame=trBL,frameround=fttt,breaklines=true,language=Perl}
\begin{lstlisting}
# 没有signature(参数)
sub print_arguments 
{
        say "Arguments:";
        for (@_)
        {
                say "\t$_";
        }
}
my @argument = qw/1 2 3 4/;
print_arguments(@argument);
\end{lstlisting}
\begin{itemize}
\item 拥有参数名称和类型的子程序
\end{itemize}

\lstset{columns=flexible,backgroundcolor=\color{background},frame=trBL,frameround=fttt,breaklines=true,language=Perl}
\begin{lstlisting}
sub distance (Int $x1, Int $y1, Int $x2, Int $y2)
{
        return sqrt ($x2-$x1)**2 + ($y2-$y1)**2;
}
say distance(3,5,0,1);
# 结果是3^2 + 4^2然后开方结果是5
\end{lstlisting}
\begin{itemize}
\item 默认参数
\end{itemize}

\lstset{columns=flexible,backgroundcolor=\color{background},frame=trBL,frameround=fttt,breaklines=true,language=Perl}
\begin{lstlisting}
sub logarithm($num,$base = 2.7183)
{
        return log($num)/log($base)
}
say logarithm(4);
# 1.38628
# 这里之提供了一个参数,所以第二个就是使用的默认的参数
say logarithm(4,2);
# 2
# 这里提供了两个参数,所以默认参数不再起作用
\end{lstlisting}
\begin{itemize}
\item 具名参数(named arguments)
\end{itemize}

\lstset{columns=flexible,backgroundcolor=\color{background},frame=trBL,frameround=fttt,breaklines=true,language=Perl}
\begin{lstlisting}
sub doit(:$when, :$what)
{
        say "doign $what at $when";
}
doit(what => "stuff",when => "once");
# doing stuff at onec
doit(:when<noon>, :what("more stuff"));
# doing more stuff at noon
\end{lstlisting}
\subsubsection{描述}
\label{sec-4-5-2}

子程序是又sub开头的关键字进行声明,可以拥有一系列的参数,
如同C,Java和其他大多数程序语言一样。
这些参数可以选择性的有类型的限制。

参数默认是只读的。但是可以通过所谓的“特性”来进行修改。

\lstset{columns=flexible,backgroundcolor=\color{background},frame=trBL,frameround=fttt,breaklines=true,language=Perl}
\begin{lstlisting}
sub try-to-reset($bar)
{
        $bar = 2; # 禁止的
}

my $x = 2;
sub reset($bar is rw)
{
        $bar=0; # 允许的
}
reset($x);
say $x; # 0

sub quox($bar is copy)
{
        $bar=3;
}
quox($x);
say $x; # is still 0
\end{lstlisting}
参数可以通过在后面添加?进行选择性是否需要这个参数。
也可以通过提供一个默认值。

\lstset{columns=flexible,backgroundcolor=\color{background},frame=trBL,frameround=fttt,breaklines=true,language=Perl}
\begin{lstlisting}
sub foo($x,$y?)
{
        if $y.defined
        {
                say "Second parameter was supplied and defined";
        }
        else
        {
                say "Don't have second parameter!";
        }
}
foo(5,6); # Secon parameter was supplied and defined
foo(5);   # Don't have second parameter!
\end{lstlisting}

\lstset{columns=flexible,backgroundcolor=\color{background},frame=trBL,frameround=fttt,breaklines=true,language=Perl}
\begin{lstlisting}
sub bar($x,$y=2*$x)
{
...
}
\end{lstlisting}
\begin{itemize}

\item 具名参数\\
\label{sec-4-5-2-1}%
当你调用一个像这样的参数的时候:my$_{\mathrm{sub}}$(\$first,\$second),
\$first参数和第一个参数是绑定的,\$second参数和第二个参数
是绑定的。这也是为什么称之为“位置参数“。

有些时候,名称比数字更好记忆,这是为什么Perl6有”具名参数的原因。

\lstset{columns=flexible,backgroundcolor=\color{background},frame=trBL,frameround=fttt,breaklines=true,language=Perl}
\begin{lstlisting}
my $r = Rectangle.new(
                x         => 100,
                y         => 200,
                height=> 23,
                width => 42,
                color => "black");
\end{lstlisting}
但你看到这种形式的东西的时候,你立马就能明白参数的意义。
为了定义一个具名参数,你只需要在参数前面添加一个冒号。

\lstset{columns=flexible,backgroundcolor=\color{background},frame=trBL,frameround=fttt,breaklines=true,language=Perl}
\begin{lstlisting}
sub area (:$width,:$height)
{
        return $width * $height;
}
area(width =>2,height=>3);
area(height =>3,width=>2);
area(:height(3),:width(2));
\end{lstlisting}
最后一个例子用了所谓的“冒号配对”语法形式。
如果仅仅有留下名字,那么会赋值为''True'',取反则会赋值为''False'':

\lstset{columns=flexible,backgroundcolor=\color{background},frame=trBL,frameround=fttt,breaklines=true,language=Perl}
\begin{lstlisting}
:draw-perimeter      # same as "draw-perimeter=>True"
:!transparent        # same as "transparent   =>False"
\end{lstlisting}
在具名参数的声明中,变量名同时用做参数的名字。你也可以用不同的名字:

\lstset{columns=flexible,backgroundcolor=\color{background},frame=trBL,frameround=fttt,breaklines=true,language=Perl}
\begin{lstlisting}
sub area (:width($w),:height($h))
{
    return $w*$h;
}
area(width=>2,height=>3);
\end{lstlisting}

\item Slurpy Parameters\\
\label{sec-4-5-2-2}%
仅仅给你的子程序命一个名字,并不意味着你事先知道子程序有多少个参数。
你可以定义所谓的slurpy parameters(在所有的正式的参数后面)可以
用所有的剩余参数。

\lstset{columns=flexible,backgroundcolor=\color{background},frame=trBL,frameround=fttt,breaklines=true,language=Perl}
\begin{lstlisting}
sub tail ($first, *@rest)
{
        say "first: $first";
        say "Rest: @rest[]";
}
tail(1,2,3,4);
# 结果是:First: 1 \n Rest:2 3 3\n";
\end{lstlisting}
具名slurpy参数是通过在哈希参数前加星号来完成的。(??????)

\lstset{columns=flexible,backgroundcolor=\color{background},frame=trBL,frameround=fttt,breaklines=true,language=Perl}
\begin{lstlisting}
sub order-meal ($name,*%extras)
{
        say "I'd like somen $name, but with a few modifications:";
        say %extras.keys.join(', ');
}
order-meal ('beef steak', :vegetarian, :well-done);
\end{lstlisting}

\item Interpolation 变量内插\\
\label{sec-4-5-2-3}%
默认情况下,数组不能内插在变量中,与Perl5不同,你可以这些写:

\lstset{columns=flexible,backgroundcolor=\color{background},frame=trBL,frameround=fttt,breaklines=true,language=Perl}
\begin{lstlisting}
sub a($scalar1,@list,$scalar2)
{
        say $scalar2;
}
my @list = "foo","bar";
a(1,@list,2); # 2
\end{lstlisting}
这也意味着你不能用列表来作为参数列表。

\lstset{columns=flexible,backgroundcolor=\color{background},frame=trBL,frameround=fttt,breaklines=true,language=Perl}
\begin{lstlisting}
my @indexes=1,4;
say "abc".substr(@indexes); # 结果是c
\end{lstlisting}
(真实发生的事情是:这第一个参数可以是Int类型,如果不是,强制转换成Int.
所以你写成''abc.''substr(@indexes.elems)是一样的。)
你可以通过使用使用潜前置的|来实现预定的功能:

\lstset{columns=flexible,backgroundcolor=\color{background},frame=trBL,frameround=fttt,breaklines=true,language=Perl}
\begin{lstlisting}
say "abcdefgh".substr(|@indexes) # bcde,same as "abcdefgh".substr(1,4);
\end{lstlisting}

\item Multi Subs 多样子程序\\
\label{sec-4-5-2-4}%
实际上,你可以用同一个子程序名称来定义多个具有不同参数列表的子程序。

\lstset{columns=flexible,backgroundcolor=\color{background},frame=trBL,frameround=fttt,breaklines=true,language=Perl}
\begin{lstlisting}
multi sub my_substr($str) {...}
multi sub my_substr($str,$start) {...}
multi sub my_substr($str,$start,$end) {...}
multi sub my_substr($str,$start,$end,$subst) {...}
\end{lstlisting}
你定义好了以后,当你调用这种子程序的时候,其中能匹配参数列表的子程序将
被调用。

这种多样子程序不仅仅能够区分参数的个数,而且能能够区分参数的类型。

\lstset{columns=flexible,backgroundcolor=\color{background},frame=trBL,frameround=fttt,breaklines=true,language=Perl}
\begin{lstlisting}
multi sub frob (Str $s) {say "Frobbing String $s"}
multi sub frob (Int $i) {say "Frobbing Integer $i"}
frob ("x");
frob (2);
# 结果
# Frobbing String x
# Frobbing Integer 2
\end{lstlisting}

\end{itemize} % ends low level
\subsubsection{Motivation真实意图}
\label{sec-4-5-3}

没人会怀疑给子程序参数一个明确名称的重要性:
更少的输入,更少的双重参数的检查,更多的自我说明性质的代码。

同时也允许有用的自省。
????
\subsection{Objects and Classes}
\label{sec-4-6}
\subsubsection{语法}
\label{sec-4-6-1}


\lstset{columns=flexible,backgroundcolor=\color{background},frame=trBL,frameround=fttt,breaklines=true,language=Perl}
\begin{lstlisting}

\end{lstlisting}
\subsection{Contexts 上下文}
\label{sec-4-7}
\subsubsection{语法}
\label{sec-4-7-1}


\lstset{columns=flexible,backgroundcolor=\color{background},frame=trBL,frameround=fttt,breaklines=true,language=Perl}
\begin{lstlisting}
my @a = <a b c>;
my $x = @a;
say $x[2];
say (~2).WHAT;
say +@a;
if @a <10 {say "short array";}
\end{lstlisting}
\subsubsection{描述}
\label{sec-4-7-2}

当你这样写的时候

\lstset{columns=flexible,backgroundcolor=\color{background},frame=trBL,frameround=fttt,breaklines=true,language=Perl}
\begin{lstlisting}
$x = @a;
\end{lstlisting}
在Perl5中,\$x 只包含@a的元素的数量。
为了保留所有的信心,你必须要用引用:\$x = \@a;

在Perl6 中,用数值变量储存数组变量,你不会丢失任何东西。
这样为上下文和更多的特殊上下文(数值,整数,和字符串)做好了准备。
Void(空)和列表上下文没有变化。
你可以强制通过语法强制转换上下文环境。

\end{document}
